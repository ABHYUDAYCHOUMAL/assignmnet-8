\documentclass{article}
\usepackage[utf8]{inputenc}
\usepackage{karnaugh-map}

% ADD TITLE HERE
\usepackage[utf8]{inputenc}

\title{K-map for g}
\author{abhtudaychoumal8 }
\date{january 2021}

\begin{document}


% ADD TITLE HERE
\title{K-map for  g}
\author{abhyuday}
\date{8 january}



\maketitle

\section{K Map}

\begin{karnaugh-map}[4][4][1][][]
    \maxterms{2,3,4,5,6,8,9,10,11,12,13,14,15}
    \minterms{0,1,7}

    \implicantedge{3}{2}{11}{10}
    \implicant{4}{13}
    \implicant{2}{10}
    \implicant{12}{10}
    
    % note: position for start of \draw is (0, Y) where Y is
    % the Y size(number of cells high) in this case Y=2
    \draw[color=black, ultra thin] (0, 4) --
    node [pos=0.7, above right, anchor=south west] {$BA$} % YOU CAN CHANGE NAME OF VAR HERE, THE $X$ IS USED FOR ITALICS
    node [pos=0.7, below left, anchor=north east] {$DC$} % SAME FOR THIS
    ++(135:1);
  
      
    \end{karnaugh-map}

\section{The boolean equation for g is:}


    g = ($\overline{D}$) (A+$\overline{B}$)($\overline{B}$+C)($\overline{C}$+B)   
    

\end{document}